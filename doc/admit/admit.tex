%% IMAGE I/O
%% 17-oct-2011 First Draft     - Peter Teuben

\documentclass[preprint]{aastex} % AASTeXv5.0
\usepackage{carma_memo}

\begin{document}
\carmamemo{999} 	

\title{ADMIT: ALMA Data MIning Toolkit}

\author{Peter Teuben}
\affil{University of Maryland}

\begin{abstract}

An overview of ADMIT (ALMA Data Mining Toolkit) V1
is given. This is the outcome of an NRAO Development
Study awarded to Mundy and Varshney (2012/2013).

\end{abstract}


\ChangeRecordBegin
\addrevision{0.1}{2013-Aug-11}{P. Teuben}{}
{Initial version transferred from image-io memo}
\ChangeRecordEnd


\section{Introduction}



\subsection{TODO}

Here's some catchphrases that need to be tracked down if they have
relevant code.

GAIA, MATADOR, AstroMed, SPLAT-VO, Herschel DP, S2Plot, AstroMD. 

\section{Line Identification}


The topic of detection and identification of spectal
lines in a complex source datacube is a complex one,
of which we offer several solutions. 

\begin{enumerate}

\item
Measuring a robust RMS in a channel (the intent being that the source is removed from the 
RMS calculation), and comparing this to the peak value in a channel gives a good 
indication of a line detection. For channels that do not contain any signal, 
the PEAK/RMS will typically be around 3 or 4, depending on the size of the map.
Note that for wide spectral windows, the RMS will vary as function of frequency,
as well as be biased where strong lines are present (or large areas of the map
contain sources). The RMS then needs to be interpolated. In NEMO the code
{\tt ccdstat planes=0} will compute plane based statistics.

\item
Inspecting a typical position-velocity (PV) cut through a cube, makes it apparent
that lines can be detected easily by eye, as their shapes are often related
(this is technically not always true, e.g. recombination lines and molecular lines have
obviously a different origin and work under different ISM conditions). Nonetheless,
by using a well known (and identified) line as a template, it can be moved up and down
along the V axis in a PV diagram, and a cross correlation then results in another method
to identify lines. The experimental codes are in NEMO as {\tt pvcorr}.


\item
A related and derived case of this
cross-correllation technique would be to compute the intensity
weighted velocity within the template, and move this curved line up and down in V and resample
the PV diagram along this shifted line, then add up all the emission along this curved
sample, and this will also be an indication of the detected lines.  Obviously this
method will generate more noise, but will typically have $\sqrt{2}$ more resolution.

\item
To balance the noise, a certain amount of smoothing to the data will always help
the identification.


\end{enumerate}

Special care has to be given to the calibration of the lines. Given the nature of
the template and the not well defined velocity of the source, the template line
must have an assumed identity and known rest frequency  ($f_{1R}$). 
With a known Doppler velocity $V_{lsr}$ its sky frequency ($f_1$) 
can then be computed.  Assuming the unknown line is measured ${\Delta f}$ away, one
can show that the rest frequency of the unknown line is given by:


$$
 f_{2R} = f_{1R} + {  {\Delta f} \over { (1-z) }}
$$

\section*{Appendix: Technical Material}

In this section we describe

\subsection{Scenarios}

Several scenarios are given, how to operate.  In this nomenclature, we use {\bf P} for
the projects, and {\bf B} for the bands (spectral windows). Each has NP projects, and 
NB bands. These bands can be independant on the frequency axis, and can contain
both line and continuum radiation. The continuum radiation is allowed to have a spectral
index (or in worst case: a true parameterized shape, e.g. black body)

\subsubsection{Pipeline run}

This would be the mode in which ASTUTE runs in the archive, i.e. from one or more spectral
windows (ALMA: NB=4, NP=1). 

\subsubsection{Simple single project archive pull}

\subsubsection{Pipeline run}

\subsubsection{Pipeline run}

\subsubsection{Pipeline run}



\subsection{Archive}

Some of ASTUTE's functionality depends on how well it hooks into the archive.

Is it going to depend on VO services.

By default ASTUTE runs and the ``big'' cubes are locally present. But if derived
product

\subsection{astute.zip}

Python has good methods to manage zip files, maintenance will be done in 
a temporary (?) subdirectory per project.

\end{document}

