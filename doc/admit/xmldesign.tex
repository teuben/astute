%$Id$
\documentclass{report}
\title{ADMIT XML Design}
\author{Marc Pound, Peter Teuben}
\date{\today}
\begin{document}
\maketitle

\section{Guiding Principles}
The XML structure should be complete enough to capture proposed use cases,
but not so restrictive that it precludes future use cases. To that end, a
limited hierarchy is desirable, with basic ``objects" that can be contained
within one another but are not {\it defined} within one another.  However, we
do need to follow what we understand to be the basic hierarchical structure
of the archive produces Project$\rightarrow$Source$\rightarrow$Bands.
We argue it is not necessary to expose the Project container to
the user, e.g. in the case of a data mining operation that spans multiple
projects. The science is in the sources not the project. To that end, the
Project container is extremely thin: it contains only the identificiation
code, everything else is inside source containers.

\subsection{Elements vs. Attributes}.  
We follow the principal that data go in elements and metadata about
elements go in attributes.   Thus rather than

\begin{verbatim}
<project name="c0123">  
\end{verbatim}

we use 

\begin{verbatim}
<project>
<name>c0123</name>
\end{verbatim}

However, Project is an example of element we may want to be read-only (ADMIT 
user cannot change it), so there we use an attribute.

\begin{verbatim}
<project readonly="true">
\end{verbatim}

\subsection{Detailed Structure}
project
    source
        band
        line 
        image
        spectrum
        task
        procedure
    


<Project>
<source>
    <name></name>
    <coordinate>
        <crvalN>
        <ctypeN>
    </coordinate>
    <equinox></equinox>
    <type> [galactic, extragalactic, etc]
</source>

<image>
   <name>
   <type> [data or thumbnail]
   <description> [moment zero, moment one, spectral cutout, full cube. ??] 
% sources in this image
   <source>A</source> % must match source names in <source> tags.
   <naxisN>
   <crpixN>
   <crvalN>
   <ctypeN>
   <statistics>
%% Can have multiple areas over which statistics are measured.
%% NB: Does casa multiple boxes print two stats or one?
        <region>
            <number>
            <blc>
            <trc>
            <startchannel>
            <endchannel>
            <mean>
            <max>
            <rms>
            <clip>
        </region>
   </statistics>
</image>
<spectrum>
   <sourcename>
   <start freq>
   <start vel>
   <statistics> </statistics> %% as above
   <channel>value</channel>
   <channel>value</channel>
   <channel>value</channel>
   <channel>value</channel> 
  %% or better as a table?
</spectrum>

%% Just rely on CASA for this. Note TASK.last files are keyword=value, not XML!
<task>
  <name>
  <parameters>
    <keyword>value</keyword>
    <keyword2>value2</keyword2>
    ...etc
  </parameters>
  <date> date/time of last run? </date>
</task>

\end{document}
