%$Id$
\documentclass{report}
\title{ADMIT XML Design}
\author{Marc Pound, Peter Teuben}
\date{\today}
\begin{document}
\maketitle

\section{Overview}

For each science project ADMIT runs a series of ``admit tasks'', which
are essentially beefed up CASA tools/tasks, and produce Basic Data
Products (BDP).  ADMIT provides a wrapper around these tasks
(CASA tasks, but with some extra baggage to make it fit within the pipeline
and have a persistent state within the pipeline as well for re-use by
the user later on).

Once users have downloaded the data, they can preview the work the pipeline
has created for them. They will also be able to re-run
selected portions of the ADMIT pipeline from either the (casapy)
commandline, or a (casa) GUI, and compare and improve upon the
pipeline produced results. ADMIT produces Basic Data Product (BDP), in
addition to the existing Alma Data Products (ADP)\footnote{Examples of
ADP's in a project are the Raw Visibility Data (in ASDM format) and
the Science Data Cubes (in FITS format) for each source and each
band}

Meta-projects can be defined, and ADMIT be used to steer and control
how to run on individual projects and sources, after which selected
results can be datamined 

\subsection{Basic Data Products}

Only the smaller portions of the BDP (Basic Data Products) can be found
in the admit.zip file, others will have to be recreated. Examples of BDP are:

\begin{enumerate}
\item
Summary of the meta data, as much as is needed for the ADMIT pipeline.
\item
Cube Statistics: a simple table of min, max, mean, median etc. for each plane
\item
Line List: a list of all lines detected in a band, but 

\end{enumerate}

\subsection{Meta Project Data Mining}

One of the novel ways in which users can interact with ADMIT is the meta project.
A series of projects (really project/source combinations)
can be selected for further work (``data mining''). 


setup a meta-project (MP) based on either all P's that have valid azip's, or use 
a complex query that goes into each Project and either selects or deselects.
Manual creation is perfectly ok as well :-)

We need some use cases here. 
    - select based on a specific line (L) present
    - select based on a source name
    - select based on a previously set user supplied keyword (on what level?)

MP needs to have some record how it was created
  
Re-run one more more MPs based on a procedure established in one P ?

Data Mining operations in MPs?

- acumulate things you did in a P into an MP (e.g. clump spectrum to gain statistics)

- extract things from a P (or do we need to distinguish P from P/S or P/S/B or P/S/L)
and stuff them in python data structures for any plotting or linked data analysis/vis
MatplotLib (?Should we use GLUE as our standard example to visualize?)


\section{Guiding Principles}
The XML structure should be complete enough to capture proposed use cases,
but not so restrictive that it precludes future use cases. To that end, a
limited hierarchy is desirable, with basic ``objects" that can be contained
within one another but are not {\it defined} within one another.  However, we
do need to follow what we understand to be the basic hierarchical structure
of the archive produces Project$\rightarrow$Source$\rightarrow$Bands.
We argue it is not necessary to expose the Project container to
the user, e.g. in the case of a data mining operation that spans multiple
projects. The science is in the sources not the project. To that end, the
Project container is extremely thin: it contains only the identificiation
code, everything else is inside source containers.

\subsection{Elements vs. Attributes}.  
We follow the principal that data go in elements and metadata about
elements go in attributes.   Thus rather than

\begin{verbatim}
<project name="c0123">  
\end{verbatim}

we use 

\begin{verbatim}
<project>
<name>c0123</name>
\end{verbatim}

However, project name is an example of element we may want to be read-only
(ADMIT user cannot change it), so there we use an attribute.

\begin{verbatim}
<project>
  <name readonly="true">c0123</name>
\end{verbatim}

\subsection{Detailed Structure}

\begin{verbatim}

project
    source
        band
        line 
        image
        spectrum
        task
        procedure
    


<Project>
<source>
    <name></name>
    <coordinate>
        <crvalN>
        <ctypeN>
    </coordinate>
    <equinox></equinox>
    <type> [galactic, extragalactic, etc]
</source>

<image>
   <name>
   <type> [data or thumbnail]
   <description> [moment zero, moment one, spectral cutout, full cube. ??] 
% sources in this image
   <source>A</source> % must match source names in <source> tags.
   <naxisN>
   <crpixN>
   <crvalN>
   <ctypeN>
   <statistics>
%% Can have multiple areas over which statistics are measured.
%% NB: Does casa multiple boxes print two stats or one?
        <region>
            <number>
            <blc>
            <trc>
            <startchannel>
            <endchannel>
            <mean>
            <max>
            <rms>
            <clip>
        </region>
   </statistics>
</image>
<spectrum>
   <sourcename>
   <start freq>
   <start vel>
   <statistics> </statistics> %% as above
   <channel>value</channel>
   <channel>value</channel>
   <channel>value</channel>
   <channel>value</channel> 
  %% or better as a table?
</spectrum>

%% Just rely on CASA for this. Note TASK.last files are keyword=value, not XML!
<task>
  <name>
  <parameters>
    <keyword>value</keyword>
    <keyword2>value2</keyword2>
    ...etc
  </parameters>
  <date> date/time of last run? </date>
</task>

\end{verbatim}

\subsection{XML Overview}
Without the cumbersome XML syntax, but simply using indentation, this is an overview
of the XML tree that admit.zip will contain

\newpage

\footnotesize
\begin{verbatim}

 Project(name)[NP]
    Summary
       <atask name=at_summary>
    Source(name)[NS]
       Summary
           ra,dec,vlsr,...
           <atask name=at_summary>
       Band(number)[NB]
           URI:im
           Summary
              FreqMin,FreqMax,FreqStep
           CubeStats
              VoTable:tab
              <atask name=at_cubestats>
              <dep>
              File:im
           LineList
              VoTable:tab
              <atask name=at_band2line>
              <dep>
                CubeStats
        LineList
           votable:tab
           <atask name=at_linemerge>
           <dep>
             band[NB].LineList
        Line(name)[NL]
           LineCube
              URI:im
              <atask name=at_reframe>
              <dep>
                LineList
           RMS (since cubestats can differ per channel)
           Mom0
              URI:im
              file:jpg
              <atask name=at_moment>
              <dep>
                LineCube
           Mom1
           Mom2
           PeakSpectrum
              VoTable:tab
              <summary>
                Peak, RMS, V0, FWHM, SdV
              <atask name=at_spectrum>
              <dep>
                LineCube
           IntegratedSpectrum
              VoTable:tab
              <summary>
                Peak, RMS, V0, FWHM, SdV
              <atask name=at_spectrum>
              <dep>
                LineCube


\end{verbatim}
\normalsize

\newpage
The dependancies might even fit in a Makefile, viz. (using the P/S/B and P/S/L hierarchy)


\footnotesize
\begin{verbatim}

p1_s1_l1_mom0:          p1_s1_l1_linecube

p1_s1_l1_linecube:      p1_s1_linelist p1_s1_b1

p1_s1_linelist:         p1_s1_b1_linelist p1_s1_b2_linelist p1_s1_b3_linelist p1_s1_b4_linelist
                        at_linemerge

p1_s1_b1_linelist:      p1_s1_b1
                        at_band2line

p1_s1_b1:               file:im
                        at_archive(p1,s1,b1)

p1_summary:
                        at_summary
        
\end{verbatim}
\normalsize


\subsection{CASA task}

The {\tt atask} will hopefully be just a casa task, of which the XML representation
looks as follows:
\footnotesize
\begin{verbatim}
<casaxml>
    <task type="function" name="foobar" category="fumbar">
        <shortdescription>
        <description>
        <input>
	    <param>
	        <description>
	        <any>
	        <value>
        <returns>
        <example>
    </task>  
</casaxml>
\end{verbatim}
\normalsize

\end{document}
