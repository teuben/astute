
\section{Access to CASA Image Cubes}

We started our study by looking at approaches for accessing large
ALMA datasets. Since image cubes, not {\it u,v} data, are the
focus of this study, we looked at the access speeds for various
operations to data cubes in a number of different formats including
the CASA tiled data format. Not too surprisingly, tiled data
access works very well for balanced cubes, but slower 
access in some of the directions for stick-like cubes (large in the
z-dimension compared to x and y).
This is detailed in the attached memo.

In the course of this work, and from discussions with CASA programmers,
several points were clear. First, CASA made the decision to follow a
tiled storage approach rather than a memory intensive model. CASA has
been effective in implementing this strategy.
With the expanding capabilities of machines and
the decrease in memory cost, it is attractive to consider tuning CASA
to be more memory based.
The tiled data storage approach of CASA is
efficient when the data are being accessed in a sequence which
follows the tiling; it can be significantly less efficient
for access which is against the grain of the tiling scheme. 
While the tile definition can be tuned to the expected access pattern
to give better performance, this is not currently utilized in CASA
as an active option.

{\it Recommendations: CASA should periodically review the memory model
for the prototype user machine and tune CASA image storage as possible to optimize
memory usage. CASA should review the usage patterns of programs to
optimized the tiling for the most commonly used programs. There may be
circumstances where it is sufficiently compute efficient to encourage users
to store a given dataset in multiple tiling formats.}

Given the evolving capability of machines and CASA, it was clear to us
that there was no strong advantage to another data storage format to increase
general compute speed at the cost additional disk usage. This conclusion
might have been different if we had been focused on a specific visualization
tool or application.

