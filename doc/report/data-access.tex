\section{Access to CASA Image Cubes}

We started our study by looking at approaches for accessing large
ALMA datasets. Since image cubes, not {\it u,v} data, are the
focus of this study, we looked at the access speeds for various
operations to data cubes in a number of different formats including
the CASA tiled data format.

FINDINGS FROM APPENDIX A

In the course of this work, and from discussions with CASA programmers,
several points were clear. First, CASA made the decision to not follow
a memory intensive model. With the expanding capabilities of machines and
the decrease in memory costs, it is attractive to consider tuning CASA
to be more memory based, but there is always a limit to the amount of
data that can be held in memory so any memory based solution is 
inherently limited. The tiled data storage approach of CASA is
fairly efficient when the data are being accessed in a sequence which
follows the tiling. A second advantage of the tile concept is that
the definition of the tiles can be tuned to the expected access pattern
to give better performance.

{\it Recommendations: CASA should periodically review the memory model
for the prototype user machine and tune CASA image storage as possible to optimize
memory usage. CASA should review the usage patterns of programs to
optimized the tiling for the most commonly used programs. There may be
circumstances where it is sufficiently compute efficient to encourage users
to store a given dataset in multiple tiling formats.}

Given the evolving capability of machines and CASA, it was not clear to us
that there was clear advantage to another data storage format to increase
general compute speed at the cost additional disk usage. This conclusion
might have been different if we had been focused on a specific visualization
tool or application.
