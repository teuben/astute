\section{Summary}

This is the final report for the ALMA Development Study on enhancing
user access to the science data in large ALMA data cubes and
creating an environment for scientific data mining. The top
goal of this study was to create a well-defined plan for
development of the software infrastructure and prototype 
key applications for enabling science with large image cubes.
The second goal was to scope the manpower and cost for achieving
a specific level of functionality which brings strong scientific
value. Based on the work done in this study, this team wrote
an ALMA Development Project Proposal to implement an XML 
infrastructure and tool set. The costs and man-power requirements
are presented in the proposal and are not reported here.

The main body of this report presents study results in the
five proposed areas. First area: in access performance, we found
that speed of access to data in CASA image format was efficient
if the data were being accessed in the optimal sequence for the
tiling; it can be inefficient if accessed against the ``grain''
of the tiling. This feature of tiling can be minimized by taking
advantage of larger memory capacities in new generations of machines
and allowing for a few different tiling choices where appropriate.

Second area: we developed a set of goals for our proposed tool set
and a technical approach for our package which we call ADMIT:
ALMA Data MIning Toolkit. We present a conceptual overview of the
design and operation and how ADMIT fits into the ALMA and CASA
environment.

Third area: we prototyped some tools for ADMIT as examples of what
can be done and as mechanisms for defining the way that ADMIT infrastructure
and tools can interact.

Fourth area: we explored a number of new algorithms that would enhance
user ability to understand the science content of large data cubes. 
We specifically discuss the overlap integral and descriptor vectors.

Fifth area: the interface between modeling and simulation data and
observational data de-emphasized as a study area so that more time
could be spend in areas two, three, and four, and the Conceptual
Design Document

The first accompanying memo summaries the results of timing measurements
of access to data by a number different programs for a number of
different data configurations. 

The second accompanying memo is a Conceptual Design Document for ADMIT.
