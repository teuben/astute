\section{Introduction}

ALMA will enable groundbreaking science over a wide range of fields. 
Exciting discoveries will be made with simple images of continuum 
emission and/or individual molecular lines. For many areas of science, 
however, such images are the tip of the iceberg; large multi-channel
data cubes and large area maps will exploit the greater capability
of ALMA.  ALMA's high sensitivity and large correlator enable
the collection of large spectral data cubes which contain a wealth of
information about the kinematic, physical, and chemical state of the 
galaxies, molecular clouds, circumstellar disks, and evolved stars under study. 
ALMA's speed can enable wide field mappings and large sample surveys.
The overall goal of this study is to outline
the tools for effectively deriving science from
large ALMA data cubes. We seek to facilitate standard 
scientific analysis and to enable new and creative ways to derive 
science from the cubes.  

The original focus areas in the study as stated in the first
section of the proposal were:
\begin{itemize}
\item ``interfaces to CASA image cubes and infrastructure for efficient, versatile access to 
these cubes in an open-source/application environment;
\item design of metadata structures that would accompany the astronomical and image data 
to enhance the utility of applications;
\item options for quantitative visualization and the speed/capability trade-offs;
\item applicability of and development paths for existing algorithms in the computer 
science community to identify and quantify structure of astronomical interest in large data cubes.
\item interfaces for intercomparison of computational models with observational data ''
\end{itemize}

These focus areas were refined over the course of the study. 
For example, the visualization area was de-emphasized in our study because 
a selected 2012 ALMA Development Study lead by Eric Rosolowsky focused
on visualization of ALMA datasets. One of our team members (Peter Teuben)
became an outside collaborator on that team; this facilitated information
exchange and reduced overlapping efforts. Second,
as a part of the study effort, we came to realize that the file size, 
Internet transfer time, and complexity of the data cubes (multiple windows
with many channels) were themselves impediments for typical users trying to 
access the science in their data. These complexities represent an even
larger barrier for scientists new to the radio field. Hence we expanded
our study in this direction.

The following sections summarize the activities and outcomes of the study
organized into sections following the above five itemized study areas.
The two appended documents provide: (A) discussion of the problem of optimizing
speed of computer access to the data and (B) an conceptual design document 
for an ``ALMA Data Mining Tool (ADMIT)''.  

This study established the groundwork for a proposal for ADMIT
that was submitted for the 2013 ALMA Call for Development Projects.

