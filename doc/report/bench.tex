\documentclass{article}
\title{BENCHMARK}
\author{Peter Teuben}
\date{\today}
\begin{document}
\maketitle

% \section*{Benchmark}

A modest benchmark was devised to get some insight in different
I/O patterns for a few packages accessible to us: CASA,
MIRIAD and NEMO. CASA uses a tiled storage manager, which in principle
can be tuned. MIRIAD has a traditional row based access, but rarely
uses memory to store full planes or cubes. NEMO is completely
memory based, but also stores the data row based (and by default
in double precision).
Three datasets of around 1GB in size were used:
a VLA Orion dataset (96 x 96 x 24012, a ``stick''),
a CARMA dataset (1500 x 1500 x 150, a ``slab''), and a
NEMO theory dataset (512 x 512 x 512, a true ``cube'').
Three operations were considered:  reading from fits,
a simple rms/mean statistics that visits the whole cube in the
most efficient way, and finally
a Hanning smooth in X, Y and Z, where possible.  For MIRIAD
the Hanning smooth test had to be skipped in X and Y, unless
an additional {\tt reorder} step would be allowed, which we did
not consider in this version of the benchmark.
It should be added that a true Hanning in MIRIAD
is the slowest possible operation, since the data is stored
row-wise.


\begin{table}[h]
\begin{center}
\begin{tabular}{|l || r r r || r r r || r r r |}
\hline
        & \multicolumn{3}{|c|}  { VLA } 
        &  \multicolumn{3}{|c|} { CARMA }
        &  \multicolumn{3}{|c|} { NEMO } \\
        & \multicolumn{3}{|c|}  { 96x96x24012 } 
        &  \multicolumn{3}{|c|} { 1500x1500x150 }
        &  \multicolumn{3}{|c|} {  12x515x512 } \\
%        & CASA  & MIRIAD & NEMO    & CASA  & MIRIAD & NEMO  & CASA & MIRIAD & NEMO \\
        & C     & M    & N         & C     & M     & N      & C    & M    & N \\
\hline
FITS    &  4.6  & 3.1  &  4.3      &  4.4  & 3.7   & 5.6    & 1.8  & 1.5  & 2.3 \\
STATS   &  3.5  & 2.6  &  2.5      &  2.7  & 3.4   & 3.8    & ?    & 1.4  & 1.4 \\
HAN-x   &  7.6  & -    &  2.6      &  2.7  & -     & 4.1    & 1.3  & -    & 1.5 \\
HAN-y   &  8.1  & -    &  2.8      &  3.1  & -     & 5.2    & 1.5  & -    & 2.5 \\
HAN-z   &  1.8  & 9.1  & 10.3      &  8.8  & 26.0  & 5.7    & 1.8  & 9.5  & 8.9 \\
\hline 
\end{tabular}
\end{center}
\caption{Comparing I/O access in a ``stick'', ``slab'' and ``cube'' like dataset. 
Times reported
are the sum of user and system time, in seconds.   C=CASA  M=MIRIAD N=NEMO(double)}
\end{table}


Not too surprisingly, the tiled access in CASA does surprisingly well for true
cubes, independant of the access direction.
But does poorly in a slab like cube, but outperforms anything else
for a stick like cube.

\end{document}




\section{Access to CASA Image Cubes}

We started our study by looking at approaches for accessing large
ALMA datasets. Since image cubes, not {\it u,v} data, are the
focus of this study, we looked at the access speeds for various
operations to data cubes in a number of different formats including
the CASA tiled data format. Not too surprisingly, tiled data 
access works very well for balanced cubes, but can have poor 
access in some of the directions for slab and stick like cubes.
This is detailed in the Appendix.

In the course of this work, and from discussions with CASA programmers,
several points were clear. First, CASA made the decision to not follow
a memory intensive model. With the expanding capabilities of machines and
the decrease in memory cost, it is attractive to consider tuning CASA
to be more memory based, but there is always a limit to the amount of
data that can be held in memory so any memory based solution is 
inherently limited. The tiled data storage approach of CASA is
fairly efficient when the data are being accessed in a sequence which
follows the tiling. A second advantage of the tile concept is that
the definition of the tiles can be tuned to the expected access pattern
to give better performance.

{\it Recommendations: CASA should periodically review the memory model
for the prototype user machine and tune CASA image storage as possible to optimize
memory usage. CASA should review the usage patterns of programs to
optimized the tiling for the most commonly used programs. There may be
circumstances where it is sufficiently compute efficient to encourage users
to store a given dataset in multiple tiling formats.}

Given the evolving capability of machines and CASA, it was not clear to us
that there was clear advantage to another data storage format to increase
general compute speed at the cost additional disk usage. This conclusion
might have been different if we had been focused on a specific visualization
tool or application.

